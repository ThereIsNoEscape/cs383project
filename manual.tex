\documentclass[12pt]{article}
\usepackage[utf8]{inputenc}

\usepackage[dvipsnames]{xcolor}
\usepackage{hyperref}

\definecolor{ggMaroon}{RGB}{72,0,108}
\definecolor{ggBlue}{RGB}{0,0,128}
\hypersetup{
	colorlinks=true,
	linkcolor=ggBlue,
	citecolor=ggBlue,
	filecolor=ggMaroon,
	urlcolor=ggMaroon
}
\urlstyle{same}



\title{RPN Calc Manual}
\author{The Group}
\date{2016 Oct 24}


\begin{document}
\setlength{\parindent}{0em}

\maketitle

\textbf{Overview} \newline
The RPN calculator adds non-signed integers together.
\vspace{5mm}

\textbf{How to use the RPN calculator}
\begin{enumerate}
	\item When you run the calculator it will ask you to enter how many numbers you would like to add together.
	\item The program will then prompt you to enter a number.
	\item The program will keep prompting you to enter a number based on how many numbers you wanted to add together.
	\item The program will then add all of the numbers together.
	\begin{itemize}
		\item Ex.\newline
		                ``Enter number count: 3`` \newline
			     ``Next Number: 1`` \newline
			     ``Next Number: 2``\newline
			     ``Next Number: 3`` \newline
			    ``6 `` \newline
	\end{itemize}
\end{enumerate}

\end{document}
